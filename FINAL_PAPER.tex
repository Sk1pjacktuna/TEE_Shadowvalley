% Options for packages loaded elsewhere
\PassOptionsToPackage{unicode}{hyperref}
\PassOptionsToPackage{hyphens}{url}
%
\documentclass[
]{article}
\usepackage{amsmath,amssymb}
\usepackage{lmodern}
\usepackage{iftex}
\ifPDFTeX
  \usepackage[T1]{fontenc}
  \usepackage[utf8]{inputenc}
  \usepackage{textcomp} % provide euro and other symbols
\else % if luatex or xetex
  \usepackage{unicode-math}
  \defaultfontfeatures{Scale=MatchLowercase}
  \defaultfontfeatures[\rmfamily]{Ligatures=TeX,Scale=1}
\fi
% Use upquote if available, for straight quotes in verbatim environments
\IfFileExists{upquote.sty}{\usepackage{upquote}}{}
\IfFileExists{microtype.sty}{% use microtype if available
  \usepackage[]{microtype}
  \UseMicrotypeSet[protrusion]{basicmath} % disable protrusion for tt fonts
}{}
\makeatletter
\@ifundefined{KOMAClassName}{% if non-KOMA class
  \IfFileExists{parskip.sty}{%
    \usepackage{parskip}
  }{% else
    \setlength{\parindent}{0pt}
    \setlength{\parskip}{6pt plus 2pt minus 1pt}}
}{% if KOMA class
  \KOMAoptions{parskip=half}}
\makeatother
\usepackage{xcolor}
\usepackage[margin=1in]{geometry}
\usepackage{graphicx}
\makeatletter
\def\maxwidth{\ifdim\Gin@nat@width>\linewidth\linewidth\else\Gin@nat@width\fi}
\def\maxheight{\ifdim\Gin@nat@height>\textheight\textheight\else\Gin@nat@height\fi}
\makeatother
% Scale images if necessary, so that they will not overflow the page
% margins by default, and it is still possible to overwrite the defaults
% using explicit options in \includegraphics[width, height, ...]{}
\setkeys{Gin}{width=\maxwidth,height=\maxheight,keepaspectratio}
% Set default figure placement to htbp
\makeatletter
\def\fps@figure{htbp}
\makeatother
\setlength{\emergencystretch}{3em} % prevent overfull lines
\providecommand{\tightlist}{%
  \setlength{\itemsep}{0pt}\setlength{\parskip}{0pt}}
\setcounter{secnumdepth}{-\maxdimen} % remove section numbering
\ifLuaTeX
  \usepackage{selnolig}  % disable illegal ligatures
\fi
\IfFileExists{bookmark.sty}{\usepackage{bookmark}}{\usepackage{hyperref}}
\IfFileExists{xurl.sty}{\usepackage{xurl}}{} % add URL line breaks if available
\urlstyle{same} % disable monospaced font for URLs
\hypersetup{
  pdftitle={Influence of maladaptive migrants on a dying population},
  pdfauthor={Florin Suter, Balz Fuchs, Felix Rentschler},
  hidelinks,
  pdfcreator={LaTeX via pandoc}}

\title{Influence of maladaptive migrants on a dying population}
\author{Florin Suter, Balz Fuchs, Felix Rentschler}
\date{}

\begin{document}
\maketitle

\hypertarget{introduction}{%
\section{Introduction}\label{introduction}}

In conservation we often see introduction of migrants as a way to save a
doomed population. There are many examples of this as for example the
rescue of the Florida panther (Puma concolor coryi) population by
introducing panthers from other regions to increase the populations gene
pool and thus rescue them from inbreeding depression.

However, in this model we thought about a diploid populations (aa) that
is maladapted to its environment and can only become a growing
population by acquiring a certain beneficial mutation (AA). Our
hypothetical scientists are attempting to save the population by
introducing migrants from a source population that doesn't carry the
beneficial allele and thus is also maladaptive. In our model we hoped to
see the impact of a migrating deleterious genotype on the extinction and
rescue probabilities of our population

\hypertarget{methods}{%
\section{Methods}\label{methods}}

For this model all coding was done using the language r. For the code
base r was sufficient however for the visualization of the results
packages such as ggplot were used to increase legibility.

\hypertarget{source-code}{%
\subsection{Source code}\label{source-code}}

The base source code used for the further coding of the model was
provided by the theoretical ecology and evolution (TEE) research group.
The code models a doomed population that will go extinct but can be
rescued by the fixation of a beneficial mutation. It simulates this for
as many generations as are input. The source code can be found on the
TEE teams platform. The rest of the code was written by us.

\hypertarget{new-code-change-the-name}{%
\subsection{New code (--\textgreater{} change the
name)}\label{new-code-change-the-name}}

\hypertarget{simulation-of-one-generation}{%
\subsubsection{Simulation of one
generation}\label{simulation-of-one-generation}}

In a first step one generation has to be simulated. As the population is
diploid there are three starting populations that are set as N\_aa =
1000, N\_Aa = 0, and N\_AA = 0. As there is a fitness difference between
the different genotypes these had to be defined aswell. For the fitness
of the different genotypes we chose fitnessaa = 0.9, fiitnessAa = 0.9,
and fitnessAA = 1.1. Both the selection coefficient and the decay rate
are contained in the fitnesses in order to keep the number of parameters
to a minimum. The next parameter is avagmigrants. This parameter
represents them mean number of migrants that enter the system in the
given generation. The genotype of these migrants is always the
maladptive aa genotype. The final parameter defined for this function is
the mutation rate mut\_rate = 0.0005. This mutation rate was chosen
arbitrarily in a way that the average number of mutants per generation
isn't over one.

The function assumes a Hardy-Weinberg equilibrium. The parts of the code
dealing with this will be described below. Some variables are mentioned
below and are not explained. For these please see the code provided in
the attachment.

\begin{enumerate}
\def\labelenumi{\arabic{enumi}.}
\item
  The total number of a and A alleles are calculated:

  \emph{a\_tot = 2*N\_aa+N\_Aa}
\item
  Number of mutations from A to a and back are calculated:

  \emph{a\_to\_A\_mut = min(rpois(1,a\_tot*mut\_rate),a\_tot)}
\item
  p and q are defined as:

  \emph{p = (a\_tot+A\_to\_a\_mut-a\_to\_A\_mut)/(2*tot\_pop)}

  \emph{q = 1-p}
\item
  Calculate frequency of the genotypes in the next generation. This is
  where selection takes place

  \emph{det\_aa\_next = (p\^{}2)*fitnessaa/avg\_fit}
\item
  Draw the true number of offspring. Note that migrants are only added
  to the aa offspring as this is the maladaptive genotype.

  \emph{offsp\_aa = rpois(1, next\_gen\_tot\_pop*det\_aa\_next) +
  aa\_migrants}

  \emph{offsp\_Aa = rpois(1, next\_gen\_tot\_pop*det\_Aa\_next)}

  \emph{offsp\_AA = rpois(1, next\_gen\_tot\_pop*det\_AA\_next)}
\end{enumerate}

\hypertarget{simulation-of-population-until-one-of-three-conditions-is-reached}{%
\subsubsection{Simulation of population until one of three conditions is
reached}\label{simulation-of-population-until-one-of-three-conditions-is-reached}}

Here the code is run until one of three conditions is reached. The first
condition is that the population reaches 1.5 times its starting size.
This is seen as successfully escaping extinction and becoming a growing,
self sustaining population. The second condition is that the population
goes extinct. As a population with a continuous stream of incoming
migrants never stays at zero individuals for long, extinction was
defined as 10* the number of migrants per generation. The final
condition that can be reached is that the population surpasses 10000
generations.

\end{document}
